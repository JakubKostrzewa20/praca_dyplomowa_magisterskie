\section{Przegląd literatury , rozwiązań oraz dostępnych baz danych}
W poniższym rozdziale został przeprowadzony przegląd literatury w zakresie wykrywania chorób roślin przy użyciu klasyfikacji zdjęć. Przedstawione zostały tutaj inne badania, przeprowadzonych przy użyciu różnych algortymów głębokiego nauczania. Opisano również zbiory danych, które zostały wykorzystane w projekcie. Są to zdjęcia lisci roślin zdrowych, jak i z różnymi chorobami, głównie gatunki hodowane przez rolników, takie jak pomidory, czy ziemniak. Na koniec przedstawiono kilka przykładów praktycznego technologii w postaci ogólnodostępnych aplikacji na telefony osobiste z systemem Android.
\subsection{Przegląd literatury}
Dzięki rozwojowi technologii nauczania maszynowego, problem rozpoznawnia roślin, czy identyfikacji chorób ich trapiących z wykorzystaniem nauczania głębokiego wzrasta na popularności. Skutkiem tego jest szeroka gamma artykułów naukowych, prac i badań poruszających ten temat. Literatura naukowa w tematyce detekcji chorób roślin wykorzystuje głównie zbiór danych PlantVillage, który zawiera rośliny rolicznie  (roz.\ref{rodzial:plantvillage}). W badaniu autorstwa Sharada P.Mohanty, David P.Hughes oraz Marcel Salathé porównują oni dwa typy architektur nauczania głębokiwego, GoogLeNet i AlexNet, w różnych konfiguracjach zmiennych oraz danych. Skupili się oni głównie na różnych wersjach tych samych danych. Dostarczyli modelom trzy zestawy tych samych zdjęć - jeden zestaw zawierał kolorowe zdjęcia, drugi zdjęcia w odcieniach szarości, a w trzecim dokonano segmentacji obiektu liścia. Wyniki badania sugerują lepszą skuteczność modelu przy użyciu pełnego kolorowego obrazu, choć różnica między trzema zbiorami danych jest minimalna \cite{sharada}. W artykule opublikowanym w AgriEngineering autorzy wykorzystuja tylko zdjecia pomidora. Następnie, po przeprowadzonej segmentacji, dokonują oni badania skalowania modelu EfficientNet co do ilości klas, które ma rozpoznać - od prostego podziału binarnego na zdrowe, bądź chore, do rozpoznawnia konkretnych chorób \cite{tomato}.  Badania te sprawdzają skuteczność różnych architektur, jednak, na co warto zwrocic uwage, mimo używania tego samego zbioru danych, sposób ich użycia, jak i przygotowania do badania różni się między autorami, co nie pozwala na uzyskanie spójnych wniosków dotyczących wyboru najlepszego typu architektury co do rozwiązania tego problemu. Warto także zwrócić uwage na fakt, że liczba architektur wykorzystana w tych badaniach jest dosyć ograniczona.

\subsection{Aplikacje}
Poniżej przedstawione aplikacje to rozwiązania ogólno dostępne w sklepie Play, wykorzystujące klasyfikacje z pomocą sztucznej inteligencji w celu identyfikacji roślin, bądź także identyfikacji trapiących ich chorób. Są one przykładem praktycznego zastosowania technologii klasyfikacji w codziennym życiu.
\subsubsection{Planta: Plant \& Garden Care}
Aplikacja skierowana do osób zajmujących się hodowlą roślin. W aplikacji znajduje się atlas roślin, w którym znajdziemy informacje na temat trudności w hodowli, cech rośliny oraz szeroki zakres informacji na temat hodowli każdego gatunku z bazy.  W ramach zakupu subskrypcji aplikacja umożliwia także identyfikacje roślin oraz dręczących ich chorób ze zdjęcia oraz planowanie i przypominanie na temat różnych czynności dotyczacych opieki nad roślina np. podlewanie czy nawożenie\cite{Planta}.
\subsubsection{Agrio - Plant diagnosis app}
Aplikacja skierowana do miłośników roślin. Pozwala na zrobienie serii zdjęć chorej rośliny, przekazanie jej aplikacji, któa następnie zadaje krótka serie pytań i na podstawie zdjeć oraz informacji z kwestionariusza, dokonuje diagnozy, jednak rozpoznaje tylko chorobę i rodzinę rośliny - nie poda informacji szczegółowych na temat gatunku ze zdjęcia. Aplikacja pozwala także, za dodatkową opłatą, na skonsultowanie się z ekspertem. Prócz funkcji diagnozy, posiada także system czat AI, gdzie użytkownik może zadawać różne pytania na temat flory. Aplikacja posiada także funkcje społecznościowe\cite{Agrio}.
\subsubsection{PlantNet Plant Identification}
Aplikacja pozwala na identyfikacje roślin na podstawie kilku metryk, np. na podstawie zdjęcia liści, kwiatu, czy ogólnego habitatu. Przedstawia także inne, prawdopdoobne gatunki pasujące do dostarczonych danych. Posiada także atas roślin, gdzie każdy gatunek z bazy ma przypisaną nazwę w wybranym z dostępnych języków, nazwę łacińską, zdjęcia, linki do źródeł informacji takich jak np. Wikipedia, oraz status gatunku w czerwonej księdze IUCN\cite{PlantNet}.
\subsection{Bazy danych}
Poniżej przedstawiono bazy danych zawierających zdjęcia liści roślin, zdrowych jak i chorych. Zbiory danych przedstawiają głównie rośliny uprawne, hodowane na całym świecie takie jak pomidor, czy kukurydza.
\subsubsection{PlantDoc: A Dataset for Visual Plant Disease Detection}
Zbiór danych CroppedPlant-Doc zawierający zdjęcia roślin zdrowych oraz chorych. Posiada ponad 2,500 zdjęć 13 gatunków roślin i 17 klas chorób. Wykorzystany został w pracy przedstawionej na konferencji  ACM India Joint International Conference on Data Science and Management of Data. Zdjęcia zostały wykonane w naturalnym środowisku, często zdjęcia prezentują także fragment oryginalnej rośliny\cite{PlantDoc}.
\subsubsection{PlantVillage Dataset}
\label{rodzial:plantvillage}
Zbiór danych  zawierający 15 klas na temat 3 roślin: ziemniak, pomidor oraz papryka. Każda roślina ma klasę przedstawiającą jej liście w zdrowym stadium oraz kilka klas przedstawiających różne choroby oraz ich objawy na strukturze liści. Zbiór zawiera łącznie ponad 20 tysięcy zdjęć. Został stworzony do konkursu na stronie AICrowd\cite{PlantVillage}.
\subsubsection{New Plant Diseases Dataset}
Zbiór danych z serwisu Kaggle zawierający ponad 87 tysięcy zdjęć roślin podzielonych w 38 różnych klas określających gatunek oraz chorobę rośliny. Dane podzielone są na zestawy treningowe oraz walidacyjne, w stosunku 80/20. To zaktualizowana wersja dataseta PlantVillage - prócz oryginalnych zdjęć zawiera dodatkowe gatunki oraz zdjęcia \cite{PlantNew}.
\clearpage