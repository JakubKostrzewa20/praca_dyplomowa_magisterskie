\newpage % Rozdziały zaczynamy od nowej strony.
\section{Zbiór danych}
\subsection{Charakterystyka zbioru danych}
Wybrany zbiór danych do wykorzystania w ramach badania to PlantVillage-Dataset. 
Jest to zaktualizowana wersja datasetu PlantVillage Dataset, który używany jest w ramach przedstawionych wcześniej badań. 
Zawiera on 54 305 zdjęć liści roślin uprawnych wykonanych w środowisku laboratoryjnym.Dataset ten posiada klasy, 
które przedstawiają zdrowe okazy różnych gatunków roślin uprawnych, oraz klasy które przedstawiają choroby trawiące te rośliny, razem 38.
\begin{figure}[h]
	\centering
	\includegraphics[width=0.2\textwidth]{img/lisc-1.png}
	\includegraphics[width=0.2\textwidth]{img/lisc-2.png}
	\includegraphics[width=0.2\textwidth]{img/lisc-3.png}
	\caption{	Przykładowe zdjęcia ze zbioru danych.}
	\label{fig:zdjecia_lisci}
\end{figure}
\subsection{Struktura klas i ich rozkład}

\subsection{Podział na zbiory treningowe i testowe}



