\newpage % Rozdziały zaczynamy od nowej strony.
\section{Zbiór danych}
\subsection{Opis zbioru}
Wybrany zbiór danych do wykorzystania w ramach badania to New Plant Diseases Dataset. Jest to zaktualizowana wersja datasetu PlantVillage Dataset, który używany jest w ramach przedstawionych wcześniej badań. Zawiera on 87 tysięcy zdjęć liści roślin uprawnych wykonanych w środowisku laboratoryjnym.Dataset ten posiada klasy, które przedstawiają zdrowe okazy różnych gatunków roślin uprawnych, oraz klasy które przedstawiają choroby trawiące te rośliny, razem 38.  Warto zwrócić uwagę na fakt, że ilości zdjęć w klasach są podobne – dataset ten, choć nie ma równomiernego rozłożenia zdjęć pomiędzy klasami, jest zbalansowany i pozwoli na uzyskanie bardziej miarodajnych wyników badania
\begin{figure}[h]
	\centering
	\includegraphics[width=0.2\textwidth]{img/lisc-1.png}
	\includegraphics[width=0.2\textwidth]{img/lisc-2.png}
	\includegraphics[width=0.2\textwidth]{img/lisc-3.png}
	\caption{	Przykładowe zdjęcia ze zbioru danych.}
	\label{fig:zdjecia_lisci}
\end{figure}
\subsection{Normalizacja i transformacja}
\subsection{Jakość zbioru danych}


