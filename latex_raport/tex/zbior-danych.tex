\newpage % Rozdziały zaczynamy od nowej strony.
\section{Zbiór danych}
\subsection{Charakterystyka zbioru danych}
Wybrany zbiór danych do wykorzystania w ramach badania to PlantVillage-Dataset. 
Jest to zaktualizowana wersja datasetu PlantVillage Dataset, który używany jest w ramach przedstawionych wcześniej badań. 
Zawiera on 54 305 zdjęć liści roślin uprawnych wykonanych w środowisku laboratoryjnym.Dataset ten posiada klasy, 
które przedstawiają zdrowe okazy różnych gatunków roślin uprawnych, oraz klasy które przedstawiają choroby trawiące te rośliny, razem 38.
\begin{figure}[H]
	\centering
	\includegraphics[width=0.2\textwidth]{img/lisc-1.png}
	\includegraphics[width=0.2\textwidth]{img/lisc-2.png}
	\includegraphics[width=0.2\textwidth]{img/lisc-3.png}
	\caption{	Przykładowe zdjęcia ze zbioru danych.}
	\label{fig:zdjecia_lisci}
\end{figure}
\subsection{Struktura klas i ich rozkład}

\begin{table}[h]
	\centering
	\begin{tabular}{|l|l|}
		\hline
		Klasa o najmniejszej ilości zdjęć & Potato\_\_\_healthy                           \\ \hline
		Klasa o największej ilości zdjęć  & Orange\_\_\_Haunglongbing\_(Citrus\_greening) \\ \hline
		Największa ilość zdjęć dla klasy  & 5507                                          \\ \hline
		Najmniejsza ilość zdjęć dla klasy & 152                                           \\ \hline
		Odchylenie standardowe            & 1254.8938177261505                            \\ \hline
		Średnia zdjęć na klasę            & 1429.078947368421                             \\ \hline
	\end{tabular}
\caption{Statystyki liczby zdjęć w zbiorze danych}
\label{tab:dataset_stats}
\end{table}
Analiza statystyczna sugeruję niezrównoważony rozkłąd zdjęć w zbiorze danych. Jak widać z tabeli \ref{tab:dataset_stats}, istnieje duża różnica ilościowa między klasą o największej ilości zdjęć - Orange\_Haunglonbging\_(Citrus\_greening), a klasą i najmniejszej ilości zdjęć - Potato\_healthy. Dodatkowo występuje duże odchylenie standardowe, bardzo bliskie średniej ilości zdjęć na klasę. Wszystko to wskazuję na potrzebe zastosowania augmentacji danych, w celu zapobiegnięcia overfittingu.
\subsection{Podział na zbiory treningowe i testowe}
W ramach badania, oryginalny zbiór danych został podzielony na trzy zbiory: treningowy, walidacyjny oraz testowy. Zbiór treningowy służący do nauki modelu zawiera 80\% wszystkich zdjęć, zbiór walidacyjny służacy do monitorowania oraz wczesnego zatrzymywania szkolenia modelu 10\%, a zbiór testowy służacy do ostatecznej oceny skuteczności modelu 10\%. Zdjęcia nie powtarzają się między zbiorami, a każda klasa występuje w każdym zbiorze, pozwalając na obiektywne wyniki. Finalnie, rozkład zdjęć w zbiorach danych wygląda następująco:
\begin{itemize}
	\item zbiór treningowy  - 43 456
	\item zbiór testowy - 5 424
	\item zbiór walidacyjny - 5 440
\end{itemize}



