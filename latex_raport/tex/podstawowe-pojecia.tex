\newpage % Rozdziały zaczynamy od nowej strony.
\section{Podstawowe pojęcia}
Poniższy rozdział zawiera wyjaśnienia wybranych pojęć, które są potem wykorzystywane w kolejnych rozdziałach pracy.  Ich zrozumienie jest potrzebne do poprawnej interpretacji opisanych w kolejnych rozdziałach metod, modeli oraz wyników przeprowadzonego badania.
\subsection{Nauczanie maszynowe}
Dziedzina sztucznej inteligencji, zajmująca się algorytmami, które pozwalają komputerom na automatyczne nauczanie poprzez ekspozycje na dane.\cite{Mitchell1997M}
\subsection{Deep learning}
Nauczanie głębokie to podzbiór uczenia maszynowego,
oparty na sztucznych sieciach neuronowych,
wykorzystujących wiele warstw, których konstrukcja
inspirowa na jest budową ludzkiego mózgu.\cite{he2016deep}
\subsection{Nauczanie nadzorowane}
Najpowszechniejsza forma deep learningu nazywana
szkoleniem nadzorowanym. Polega on na dostarczeniu
modelowi zbioru danych z etykietami klas. Model tworzy
funkcje celu, gdzie na wejściu otrzymuje dane, a na wyjściu otrzymuje etykietę. Podczas szkolenia model modyfikuje swoje wewnętrzne parametry tak, aby funkcja mogła dostarczonej mu danie przypisać odpowiednią etykietę.\cite{lecun2015deep}
\subsection{Klasyfikacja}
Analiza obrazów na podstawie ich tresci w celu przyporządkowania ich do wcześniej zdefiniowanyuch klas.
\subsection{Konwolucyjne sieci neuronowe - CNN}
Konwolucyjne sieci neuronowe to architektura deep
learning, opiera ją ca się na czterech kluczowych konceptach:połączenia lokalne, współdzielone wagi, grupowanie i wykorzystywanie wielu warstw. Typowa architektura modelu CNN zawiera w sobie: warstwy konwolucyjne, które dokonują ekstrakcji cech; warstwy poolingowe, które dokonują zmniejszenia wymiarów inputu, scalając podobne cechy w jedną , ułatwiając obliczenia ; warstwy aktywacyjnej
oraz warstw w pełni połączonych, gdzie na podstawie
nauczonych parametrów z poprzednich warstw, dokonywana
jest klasyfikacja danych. Na poniższym obrazku przedstawiona jest kontrukcja takiej sieci.\cite{lecun2015deep}
\begin{figure}[h]
	\centering
	\includegraphics[width=0.6\textwidth]{img/cnn.png}
	\caption{Przedstawienie budowy CNN.\cite{lecun2015deep}}
	\label{fig:cnn_rys}
\end{figure}
\subsection{Transformer wizyjny - ViT}
Transformer wizyjny to architektura działająca oddmienie od CNN. Obraz dostarczony do modelu jest dzielony na kawałki, tak zwane patche, o z góry narzuconym rozmiarze. Każdy z patchy jest płaszczony, a następnie mapowany do wektora. Dodawana do wektora jest pozycje patchy, a otrzymany wynik jest przepuszczany przez enkoder transformera. Przegląd modelu znajduję się na  . Transformer wizyjny, dzięki swojej unikalnej strukturze, lepiej tworzy powiązania globalne między różnymi elementami na obrazku niż klasyczne podejście CNN.\cite{dosovitskiy2020image}
\begin{figure}[h]
	\centering
	\includegraphics[width=0.6\textwidth]{img/vit.png}
	\caption{Przedstawienie budowy oraz działania ViT.\cite{dosovitskiy2020image}}
	\label{fig:vit_rys}
\end{figure}
\subsection{Balans klas}
Balans klas odnosi się do rozkładu danych w ramach zbioru danych używanego w szkoleniu. Jeśli istnieją spore różnice w ilościach danych, między klasami, zbiór jest określany wtedy jako niezbalansowany, analogicznie, jeśli dane są równomiernie rozłożone to zbiór można określić jako zbalansowany.
\subsection{Rodzaje zbiorów}
Dane używane do procesu szkolenia modelu oraz jego ewaluacji, dzielone są na trzy zbiory danych, każde posiadające unikalne zdjęcia oraz rolę w całym procesie:
 \begin{itemize}
 	\item Zbiór treningowy (training set) - zbiór danych wykorzystywany przez model do szkolenia;
 	\item Zbiór walidacyjny (validation set) - zbiór używany do w wyznaczenia optymalnych hiperparametrów modelu;
 	\item Zbiór testowy (test set) - zbiór danych wykorzystywanych do przetestowania wyszkolonego modelu;
 \end{itemize}
\subsection{Generalizacja}
Zdolność modelu do klasyfikacji nie widzianej wcześniej dany, na podstawie przeprowadzonego wcześniej treningu.
\subsection{Przeuczenie}
Zjawisko występujące, kiedy model osiąga wysokie wyniki na zbiorze treningowym, jednak kiedy otrzymuje on wcześniej niewidziane dane, nie jest on wstanie poprawnie dokonać klasyfikacji.